\documentclass{article}
\usepackage{graphicx} % Required for inserting images
\usepackage{amsmath}
\usepackage{amsfonts}

\title{MA 294 Formula Sheet}
\author{Xinyang(Tommy) Cheng}
\date{February 2024}

\begin{document}

\maketitle

\section{Chapter One}
\subsection{De Morgan's Law}
\begin{enumerate}
\item $(A \cup B)'=A' \cap B'$
\item $(A \cap B)'=A' \cup B'$
\end{enumerate}
\subsection{Cartesian Product}
$$A \times B = {(a,b): a\in A, b\in B}$$

\subsubsection{Relations}
\text{Subsets of the Cartesian Product are called relations.}

\subsubsection{Functions (Mapping)}
$$f\subset A \times B$$
\text{Function is a special type of relation where} \textbf{each element $a\in A$ has a unique element $b\in B$ such that $(a,b)\in f$}.
\text{The set $A$ is the domain of $f$ and $f(A)$ is the image/range.}
\paragraph{Surjective (Onto)}
\subparagraph{if there exists an $a\in A$ for each $b\in B$ such that $f(a)=b$; for each $b\in B$ there is at least one corresponding $a\in A$}
\paragraph{Injective (One-to-one)}
\subparagraph{if $f(a_1)=f(a_2)$ implies $a_1=a_2$; for each $b\in B$ there is at most one corresponding $a\in A$}
\paragraph{Bijective} a map that is both surjective and injective

\subsection{Composition Theorem}
$$(g\circ f)(x)=g(f(x))$$
\begin{itemize}
    \item The composition of msppings is associative.
        $$(h\circ g)\circ f=h\circ (g\circ f)$$
    \item If $f$ and $g$ are both one-to-one, the mapping $g\circ f$ is one-to-one.
    \item If $f$ and $g$ are both onto, the mapping $g\circ f$ is onto.
    \item If $f$ and $g$ are bijective, the mapping $g\circ f$ is bijective.
\end{itemize}

\subsection{Identity Mapping}
for any set $S$ \\
notation: $id$ or $id_S$; \\$id(s)=s \;\forall s\in S$
\subsubsection{Inverse Mapping}
for $f: A\rightarrow B$:\\
$g: B\rightarrow A$ is the inverse mapping of it\\
$$g\circ f=id_A; f\circ g=id_B$$
A map is said to be invertible if it has an inverse.

\subsection{Equivalence Relations}
For a set $X$, a relation $R\subset X\times X$ that is:
\begin{itemize}
    \item reflexive: $(x,x)\in R \;\forall x\in X$
    \item symmetric: $(x,y)\in R$ implies $ (y,x)\in R$
    \item transitive: $(x,y)\in R, (y,z)\in R$ impiles $(x,z)\in R$
\end{itemize}
The relation is said to be an equivalence relation, written as $x\sim y$.

\subsection{Partition}
a partition of a set $X$ is a collection of nonempty sets $X_1, X_2,...$ such that:
\begin{itemize}
    \item $X_i \cap X_j = \emptyset$ for $i\neq j$
    \item $\bigcup_i X_i=X$
\end{itemize}
\subsubsection{Equivalence Class}
$$[x]=\{y\in X: x\sim y\}$$
\section{Chapter Two}
\subsection{Mathematical Induction}
base case+inductive step\\
recall CAS CS 131
\subsection{Division Algorithm}
$$a=bq+r$$
\begin{itemize}
    \item $a,b,q,r \in \mathbb{Z}$
    \item $0\leq r<b$
\end{itemize}
If $b=ak$ for some integers $k$, we write $a|b$
\subsubsection{Common Divisor}
an integer $d$ is a common divisor if $d|a$ and $d|b$
\subsubsection{Greatest Common Divisor}
a positive integer $d$ such that $d$ is a common divisor of integers $a$ and $b$; for any other common divisors $d'$ of $a$ and $b$, $d'|d$\\\\
notation: $d=gcd(a,b)$\\
$gcd(a,b)=ar+bs$ for some integers $r,s$\\

\textbf{Relatively Prime}: integers $a$ and $b$ are relatively prime if $gcd(a,b)=1$\\

\section{Chapter Three}
\subsection{Integer Equivalence Classes and Symmetries}
\begin{itemize}
    \item $\mathrm{Z}_n=\{[0],[1],[2],...,[n-1]\}$ as a set of equivalence classes where arithmetic is $mod_n$
    \item $a\equiv b(mod \: n)$ if $n|a-b$
\end{itemize}
\paragraph{Symmetry}
\subsection{Definitions and Examples}
\subsubsection{Group}
A group $(G,\circ)$ is a set $G$ together with a law of composition $\circ$ that satisfies the following axioms:
\begin{itemize}
    \item the law of composition is associative: 
    $(a\circ b)\circ c=a\circ (b\circ c)$ for $a,b,c\in G$
    \item there exists an element $e\in G$ as the identity element, such that for any element $a\in G$, $e\circ a=a\circ e=a$
    \item for each element $a\in G$, there exists an inverse element in $G$ denoted by $a^{-1}\in G$ such that $a\circ a^{-1}=a^{-1}\circ a=e$
\end{itemize}
A group is said to be \textbf{abelian} if $a\circ b=b\circ a \;\forall a,b\in G$, or else being \textbf{nonabelian}.
\\Laws of Exponents [$\forall g,h\in G$]\\
\begin{itemize}
    \item $g^{m}g^{n}=g^{m+n}\;\forall m,n\in \mathbb{Z}$
    \item $(g^m)^n=g^{mn}\;\forall m,n\in \mathbb{Z}$
    \item $(gh)^n=(h^{-1}g^{-1})^{-n}\;\forall n\in \mathbb{Z}$
    \item if $G$ is abelian, $(gh)^n=g^{n}h^{n}\;\forall n\in\mathbb{Z}$
\end{itemize}
\subsubsection{Subgroup}
A subgroup $H$ of a group $G$ needs to be
\begin{enumerate}
    \item a subset $H$ of $G$
    \item the group operation of $G$ is restricted to $H$
    \item $H$ itself is a group (meet the conditions)
\end{enumerate}
Every group with at least two elements will always have at least two subgroups: the subgroups with only the identity element and the entire group itself.
\begin{itemize}
    \item \textbf{Trivial Subgroup:} $H=\{e\}$
    \item \textbf{Proper Subgroup:} a subgroup that is a proper subset
\end{itemize}

-\\A subset $H$ of $G$ is a subgroup if and only if it satisfies the following conditions:
\begin{itemize}
    \item The identity $e$ of $G$ is in $H$.
    \item If $h_1,h_2\in H$, then $h_1 h_2\in H$.
    \item If $h\in H$, then $h^{-1}\in H$.
\end{itemize}

\section{Chapter Four}
\subsection{Cyclic Subgroups}
Let $G$ be a group and $a$ be any element in $G$. Then the set $<a>=\{a^{n}: n\in \mathbb{Z}\}$ is a subgroup of $G$. \\
Furthermore, $<a>$ is the smallest subgroup of that contains $a$.
\begin{itemize}
    \item For $a\in G$, we call $<a>$ the \textbf{cyclic subgroup} generated by $a$.
    \item If $G$ contains some element $a$ such that $G=<a>$, then $G$ is a \textbf{cyclic group}. In this case, $\mathbf{a}$ \textbf{is a generator of }$\mathbf{G}$.
    \item If $a$ is an element of a group $G$, we define the \textbf{order} of $a$ to be the smallest positive integer $n$ such that $a^n=e$, and we write $|a|=n$. If there is no such integer $n$, we say that the order of $a$ is infinite and write $|a|=\infty$ to denote the order of $a$.
\end{itemize}
\subsubsection{Theorems}
\begin{enumerate}
    \item Every cyclic group is abelian.
    \item Every subgroup of a cyclic group is cyclic.
    \item The subgroups of $\mathbb{Z}$ are exactly $n\mathbb{Z}$ for $n=0,1,2,3,...$.
    \item Let $G$ be a cyclic group of order $n$ and suppose that $a$ is a generator for $G$. Then $a^{k}=e$ if and only if $n$ divides $k$.
    \item Let $G$ be a cyclic group of order $n$ and suppose that $a$ is a generator of the group. If $b=a^k$, then the order of $b$ is $n/d$, where $d=gcd(k, n)$.
    \item Generators of $\mathbb{Z}_n$ are the integers $r$ such that $1\leq r<n$ and $gcd(n, r)=1$.
\end{enumerate}

\section{Chapter Five}
\subsection{Definitions \& Notation}
\textbf{Theorem:} The \textbf{symmetric group} on $n$ letters, $S_n$, is a group with $n!$ elements, where the binary operation is the composition of maps.
\\\\\\\\
\subsubsection{Cycle Notation}
A permutation $\sigma \in S_X$ is a \textbf{cycle of length} $k$ if there exists elements $a_1, a_2, ..., a_k \in X$ such that
$$\sigma(a_1)=a_2$$
$$\sigma(a_2)=a_3$$
$$...$$
$$\sigma(a_k)=a_1$$
and $\sigma(x)=x$ for all other $x\in X$. We write $(a_1 a_2 a_3 ... a_k)$ to denote the cycle $\sigma$.
\begin{itemize}
    \item A subgroup of $S_n$ is called a permutation group.
    \item Two cycles in $S_n$, $\sigma=(a_1 a_2 ... a_k)$ and $\tau=(b_1 b_2 ... b_l)$ are disjoint if $a_i \neq b_i$ for all $i$ and $j$.
    \item Let $\sigma$ and $\tau$ be two disjoint cycles in $S_n$. Then $\sigma\tau=\tau\sigma$.
    \item Every permutation in $S_n$ can be written as the product of disjoint cycles. 
    \item A cycle of length 2 is a \textbf{transportation}.
    \item If the identity $id$ is written as the product of $r$ transpositions, then $r$ is an even number.
    \item If a permutation $\sigma$ can be expressed as the product of an even numer of transpositions, then any other product of transpositions equaling must also contain an even number of transpositions.
    \item The set of even permutations $A_n$ is a subgroup of $S_n$.
\end{itemize}

\end{document}
